\documentclass[12pt]{article}
\usepackage{fullpage}

\usepackage{amsmath}
\usepackage{amssymb}
\usepackage{amsthm, algorithm}
\usepackage{hyperref}
\usepackage{color}

\newtheorem{problem}{Problem}
\newtheorem{theorem}{Theorem}[section]
\newtheorem{prop}[theorem]{Proposition}
\newtheorem{corollary}{Corollary}[theorem]
\newtheorem{remark}{Remark}[theorem]
\newtheorem{lemma}[theorem]{Lemma}
\newtheorem{definition}[theorem]{Definition}
\newtheorem{observation}[theorem]{Observation}
\newtheorem{hole}{Hole}[theorem]

\newcommand{\cupdot}{\mathbin{\mathaccent\cdot\cup}}

\newcommand{\tr}{\mbox{Trace}}
\newcommand\prob[2]{\mbox{Pr}_{#1}\left[ #2 \right]}
\newcommand\expec[2]{{\mathbb{E}}_{#1}\left[ #2 \right]}
\newcommand\var[2]{\mbox{\bf Var}_{#1}\left[ #2 \right]}


\newcommand\Ctil{\widetilde{\mathit{C}}}
\newcommand\Otil{\widetilde{\mathit{O}}}

\newcommand\Ccal{\mathcal{C}}
\newcommand\Hcal{\mathcal{H}}

\newcommand{\defeq}{\stackrel{\textup{def}}{=}}
\renewcommand\AA{\boldsymbol{\mathit{A}}}
\newcommand\DD{\boldsymbol{\mathit{D}}}
\newcommand\MM{\boldsymbol{\mathit{M}}}
\newcommand\MMcal{\boldsymbol{\mathcal{M}}}
\newcommand\MMbar{\boldsymbol{\overline{\mathit{M}}}}
\newcommand\MMhat{\boldsymbol{\widehat{\mathit{M}}}}
\newcommand\II{\boldsymbol{\mathit{I}}}
\newcommand\LL{\boldsymbol{\mathit{L}}}
\newcommand\LLtil{\widetilde{\boldsymbol{L}}}

\newcommand\cchi{\boldsymbol{\chi}}

\newcommand\simuniform{{\sim_{{\rm uniform}}}}

\newcommand\Ical{\mathcal{I}}

\newcommand\yhat{\widehat{y}}

\newcommand\AbsMatrix[1]{\mbox{Abs}_{2}\left| #1 \right|}


\newcommand\dd{\boldsymbol{\mathit{d}}}
\newcommand\rr{\boldsymbol{\mathit{r}}}
\newcommand\ww{\boldsymbol{\mathit{w}}}
\newcommand\xx{\boldsymbol{\mathit{x}}}
\newcommand\yy{\boldsymbol{\mathit{y}}}
\newcommand\eps{\varepsilon}


\newcommand\PPi{\boldsymbol{\mathit{\Pi}}}
\newcommand\one{\vec{1}}

\newcommand{\todo}[1]{{\bf \color{red} TODO: #1}}
\newcommand{\richard}[1]{{\bf \color{green} RICHARD: #1}}
\newcommand{\junxing}[1]{{\bf \color{green} JUNXING: #1}}
\newcommand{\saurabh}[1]{{\bf \color{green} SAURABH: #1}}
\newcommand{\sushant}[1]{{\bf \color{green} SUSHANT: #1}}
\newcommand{\tim}[1]{{\bf \color{red} TIMOTHY: #1}}

\title{
  Effective Resistances is a Hypermetric
}
\author{Timothy Chu\\
  Carnegie Mellon \\
    \texttt{tzchu@andrew.cmu.edu}
}
\date{\today}
\begin{document}

\maketitle
\thispagestyle{empty}

\begin{abstract} The effective resistance is a widely used metric.
It is known to be of negative type.

Negative type metrics are known to contain a wide class of metrics. In
particular, it is known that L2 is contained inside L1, which is
contained inside the space of Hypermetrics. The space of Hypermetrics is
in turn contained inside the space of negative type metrics.

We aim to precisely determine how effective resistance metrics fit into
this heirarchy of metrics: which metric spaces contain all effective
resistance metrics, and which metrics are contained inside the space of
effective resistance metrics?

In this paper, we show that L2 is not contained in the effective
resistance metrics, and that the effective resistance metrics are
contained in the hypermetrics.
To do this, we introduce both an easily computable criterion for
determining whether a given distance can be induced by an effective
resistance distance, and a generalization of the Rayleigh
Monotonicity law. These two may be of independent interest when
studying properties of Effective Resistances.

\end{abstract}
\section{Introduction}
\subsection{Effective Resistances}
\begin{enumerate}
\item Effective Resistances are a metric.
\item They're metrics of negative type, a fact useful for computing them quickly.
(Spielman Srivastava).
\item They measure expected commute time, the probability that any given
edge is contained in a random spanning tree, and more.
\item Question: What more can we say about them, with respect to
well-studied metric theory?
\end{enumerate}

\subsection{Negative Type Metrics}
\begin{enumerate}
\item Negative type metrics are known to contain: L2, L1, Hypermetrics,
  ...
\item Theory of Isometric Embeddings is of independent interest.
\item Known that $l_2 \subset l_1 \subset HyperMetrics \subset NegType$.
\item Question: Where does effective resistance fit in? 
\item Conjecture: Effective Resistance is contained inside $l_1$.
\item Theorem: Effective Resistances are contained in Hypermetrics, are
not contained in $l_2$, and $l_2$ does not contain the effective
resistance distance.
\end{enumerate}
\subsection{Hypermetrics}
\begin{enumerate}
\item Polygonal inequalities satisfied.
\item Mostly of mathematical interest.
\item A full classification of Hypermetrics is known (also: l1 metrics,
    negative type metrics.)
\end{enumerate}



\section{Overview}
\section{A Quadratic Time Criterion for Effective Resistance.}
\section{A Generalization of Rayleigh's Monotonicity}
\section{Effective Resistance Obeys the Hypermetric Inequalities.}
\section{Open Problems}
Hello.
\end{document}
